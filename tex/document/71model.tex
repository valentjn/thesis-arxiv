\section{Continuum-Mechanical Model of the Upper Limb}
\label{sec:71model}

\minitoc{71mm}{6}

\noindent
In the following, we first discuss the state of the art
in biomechanical modeling.
Then, we address details of the model of the human upper limb.
For convenience, the most relevant symbols are listed in
\cref{tbl:glossaryMusculoskeletal}.

\begin{table}
  \setnumberoftableheaderrows{0}%
  \newcommand*{\pnst}[1]{%
    \printnotationsymbol{#1}%
    \vphantom{\printnotationsymbol{\equielbang}$(\cdot)^{\reference}$}&%
    \printnotationtext{#1}%
  }%
  \begin{tabular}{%
      >{\kern\tabcolsep}=l+l<{\kern5mm}+l+l<{\kern5mm}+l+l<{\kern\tabcolsep}%
    }
    \toprulec
    \pnst{\forceT}& \pnst{\armT}&       \pnst{\actT}\\
    \pnst{\forceB}& \pnst{\armB}&       \pnst{\actB}\\
    \pnst{\forceL}& \pnst{\armL}&       \pnst{\moment}\\
    \pnst{\elbang}& \pnst{\tarelbang}&  \pnst{\equielbang}\\
    \pnst{t}&
    $(\cdot)^{\sparse}$&Sparse grid solution&
    $(\cdot)^{\reference}$&Reference solution\\
    \bottomrulec
  \end{tabular}%
  \caption[Glossary for musculoskeletal models]{%
    Glossary of the notation for musculoskeletal models.%
  }%
  \label{tbl:glossaryMusculoskeletal}%
\end{table}



\subsection{Continuum-Mechanical Musculoskeletal Models}
\label{sec:711models}

\paragraph{%
  Limitations of classical models and
  benefits of continuum-mechanical models%
}

Due to the simplicity of classical lumped-parameter models,
their degree of realism is limited.
Without any modifications,
lumped-parameter models are not able to represent
detailed heterogeneous material characteristics or non-trivial
muscle force paths \cite{Roehrle16Two}.

The exploitation of continuum-mechani\-cal constitutive laws
for musculoskeletal models is a more recent development \cite{Roehrle16Two}.
The resulting models are capable to model spatial quantities
such as complex muscle fiber field architectures,
local activation principles, complex muscle geometries, or contact mechanics
\multicite{Roehrle16Two,Valentin18Gradient}.
Most of the existing work only treats single skeletal muscles in isolation
\multicite{Lemos05Modeling,Sharafi11Strains,Heidlauf14Multiscale}.
The model used in this thesis
(which is the same model as in
\multicite{Sprenger15Continuum,Roehrle16Two,Valentin18Gradient})
aims at studying the interplay of multiple muscles and bones.

\paragraph{Overdetermined antagonistic systems}

Musculoskeletal systems are typically overdetermined \cite{Roehrle16Two}.
This means that the number of muscles that act on a specific joint
is usually larger than the number of the joint's degrees of freedom.
For instance, in a simple model of the human upper limb,
there are two antagonistic muscles
(i.e., muscles that work against each other), namely triceps and biceps,
but only one joint angle at the elbow.
Mathematically speaking, a single muscle suffices to attain
a large range of elbow angles
that are possible with an antagonistic muscle pair.
However, the usage of two muscles enables faster movements and
allows for abrupt changes of direction.

The overdetermination of most musculoskeletal models
implies that additional conditions have to be imposed in order
to obtain unique solutions.
There exist various types of such conditions,
for instance, minimal control effort, minimal control change, and
minimal kinematic energy \cite{Valentin18Gradient}.
The idea behind these conditions is that the human body
tries to minimize the energy effort that is associated with
all types of motion.

\paragraph{Forward and inverse dynamics}

Musculoskeletal simulations are usually based on
either forward dynamics or inverse dynamics \cite{Valentin18Gradient}.
\term{Forward-dynamic approaches} use activation parameters
for the muscles as the input and simulate the corresponding motion
as the output.
This requires that we know the muscle forces
(depending on the activation levels) beforehand.
For example, one can prescribe activation levels of facial muscles
to achieve specific facial expressions \cite{Wu13Modelling}.
In contrast, \term{inverse-dynamic approaches}
use experimental motion data as the input
to estimate the muscle forces as the output \cite{Roehrle16Two}.
With inverse-dynamic simulations,
one can investigate the wrapping of muscles
around the knee joint \cite{Fernandez05Anatomically} or
visualize the motion of skin \cite{Lee09Comprehensive}, for instance.



\subsection{Details of the Human Upper Limb Model}
\label{sec:712details}

As shown in \cref{fig:raisingArm},
our model of the human upper limb consists
of the three bones humerus, ulna, and radius,
the elbow joint with one degree of freedom, and
the antagonistic muscle pair of triceps brachii and biceps brachii
\cite{Valentin18Gradient}.
The bones are rigid bodies and the muscle-tendon complex
is simulated with a continuum-mechanical approach.
This implies that the muscles deform when they contract.

\begin{figure}
  \begin{tikzpicture}[
    dashed/.style={dash pattern=on 2pt off 2pt},
    contour/.style={line width=3pt,draw=mittelblau!10},
  ]
    \newcommand*{\myhspace}{30.34mm}
    \newcommand*{\myscale}{0.14}
    \newcommand*{\mytricepsarmstart}{0.1}
    \newcommand*{\mybicepsstart}{0.2}
    \newcommand*{\myradius}{9mm}
    \newcommand*{\mysmallradius}{2mm}
    
    \foreach \myi in {1,...,5} {
      \node[anchor=north west] at ({(\myi-1)*\myhspace},0mm) {%
        \includegraphics[scale=\myscale]{upperLimb_\myi}%
      };
      \node[anchor=north west] at ({(\myi-1)*\myhspace+11mm},0mm) {%
        \pgfmathparse{10+35*(\myi-1)}%
        $\theta = \ang[
          round-mode=places,round-precision=0,
        ]{\pgfmathresult}$%
      };
    }
    
    \begin{scope}[shift={({(3-1)*\myhspace},0mm)}]
      \coordinate (joint)               at (8mm,-30mm);
      \coordinate (loadForceStart)      at (30.6mm,-35mm);
      \coordinate (loadForceEnd)        at (30.6mm,-45mm);
      \coordinate (tricepsForceStart)   at (5.7mm,-31.9mm);
      \coordinate (tricepsForceEnd)     at (4mm,-15mm);
      \coordinate (tricepsArmStart)     at (
        ${(1-\mytricepsarmstart)}*(joint) +
        \mytricepsarmstart*(loadForceStart)$
      );
      \coordinate (bicepsForceStart)    at (
        ${(1-\mybicepsstart)}*(joint) + \mybicepsstart*(loadForceStart)$
      );
      \coordinate (bicepsForceEnd)      at (11.5mm,-10mm);
      \coordinate (horizontalDashedEnd) at ($(loadForceEnd) + (0mm,3mm)$);
      \coordinate (verticalDashedEnd)   at ($(joint) - (0mm,15mm)$);
      \pgfmathanglebetweenpoints{
        \pgfpointanchor{joint}{center}
      }{
        \pgfpointanchor{loadForceStart}{center}
      }
      \let\myelbowangle\pgfmathresult
      \coordinate (arcStart) at ($(joint) - (0mm,\myradius)$);
      \coordinate (arcEnd) at (
        $(joint) + (
          {\myradius*cos(\myelbowangle)},
          {\myradius*sin(\myelbowangle)}
        )$
      );
      
      \makeatletter
      \newcommand*{\drawleverarm}[5][\@nil]{
        \def\myoptarg{#1}
        \ifx\myoptarg\@nnil
          \def\mypointsize{1pt}
          \def\myfill{C1}
          \def\myoptarg{}
        \else
          \def\mypointsize{2pt}
          \def\myfill{mittelblau!10}
        \fi
        \coordinate (leverArmBase) at ($(#3)!(#2)!(#4)$);
        \draw[dashed,draw=C1,\myoptarg] (#2) -- (leverArmBase);
        \pgfmathanglebetweenpoints{
          \pgfpointanchor{#2}{center}
        }{
          \pgfpointanchor{leverArmBase}{center}
        }
        \let\myangle\pgfmathresult
        \pgfmathparse{\myangle #5 180}
        \let\mystartangle\pgfmathresult
        \pgfmathparse{\myangle #5 90}
        \let\myendangle\pgfmathresult
        \centerarc[draw=C1,\myoptarg](leverArmBase)(
          \mystartangle:\myendangle:\mysmallradius
        );
        \fill[draw=none,fill=\myfill] (
          $(leverArmBase) + (
            {0.62*\mysmallradius*cos((\mystartangle+\myendangle)/2)},
            {0.62*\mysmallradius*sin((\mystartangle+\myendangle)/2)}
          )$
        ) circle (\mypointsize);
      }
      \makeatother
      
      \draw[contour] (joint) -- (loadForceStart);
      \draw[contour] (tricepsArmStart) -- (tricepsForceStart);
      
      \draw[dashed,contour] (joint) -- (verticalDashedEnd);
      \draw[dashed,contour] (
        $(joint)!(horizontalDashedEnd)!(verticalDashedEnd)$
      ) -- (horizontalDashedEnd);
      \draw[dashed,contour] (joint) -- (
        $(tricepsForceStart)!(joint)!(tricepsForceEnd)$
      );
      \draw[dashed,contour] (joint) -- (
        $(bicepsForceStart)!(joint)!(bicepsForceEnd)$
      );
      \centerarc[dashed,contour](joint)(270:\myelbowangle:\myradius);
      
      \drawleverarm[contour]{horizontalDashedEnd}{joint}{verticalDashedEnd}{-}
      \drawleverarm[contour]{joint}{tricepsForceStart}{tricepsForceEnd}{-}
      \drawleverarm[contour]{joint}{bicepsForceStart}{bicepsForceEnd}{+}
      
      \draw[
        ->,contour,>={Stealth[width=9pt,length=9pt]},
      ] (tricepsForceStart) -- (
        $(tricepsForceEnd)!-1.5pt!(tricepsForceStart)$
      );
      \draw[
        ->,contour,>={Stealth[width=9pt,length=9pt]},
      ] (bicepsForceStart) -- (
        $(bicepsForceEnd)!-1.5pt!(bicepsForceStart)$
      );
      \draw[
        ->,contour,>={Stealth[width=9pt,length=9pt]},
      ] (loadForceStart) -- (
        $(loadForceEnd)!-1.5pt!(loadForceStart)$
      );
      
      \draw (joint) -- (loadForceStart);
      \draw (tricepsArmStart) -- (tricepsForceStart);
      
      \draw[dashed] (joint) -- (verticalDashedEnd);
      \centerarc[dashed](joint)(270:\myelbowangle:\myradius);
      
      \drawleverarm{horizontalDashedEnd}{joint}{verticalDashedEnd}{-}
      \drawleverarm{joint}{tricepsForceStart}{tricepsForceEnd}{-}
      \drawleverarm{joint}{bicepsForceStart}{bicepsForceEnd}{+}
      
      \draw[->,draw=C0] (tricepsForceStart) -- (tricepsForceEnd);
      \draw[->,draw=C0] (bicepsForceStart) -- (bicepsForceEnd);
      \draw[->,draw=C0] (loadForceStart) -- (loadForceEnd);
      
      \fill[draw=none,fill=black] (joint) circle (2pt);
      
      \node at (
        $(joint) + (
          {0.62*\myradius*cos((270+\myelbowangle)/2)},
          {0.62*\myradius*sin((270+\myelbowangle)/2)}
        )$
      ) {\contour{mittelblau!10}{$\elbang$}};
      
      \node[anchor=south] at (
        $0.5*(joint) +
        0.5*(tricepsForceStart)!(joint)!(tricepsForceEnd) +
        (-1mm,1mm)$
      ) {\contour{mittelblau!10}{\textcolor{C1}{$\armT$}}};
      \node[anchor=south] at (
        $0.5*(joint) +
        0.5*(bicepsForceStart)!(joint)!(bicepsForceEnd) +
        (1mm,1mm)$
      ) {\contour{mittelblau!10}{\textcolor{C1}{$\armB$}}};
      \node[anchor=north] at (
        $0.5*(horizontalDashedEnd) +
        0.5*(joint)!(horizontalDashedEnd)!(verticalDashedEnd) +
        (0mm,-1mm)$
      ) {\contour{mittelblau!10}{\textcolor{C1}{$\armL$}}};
      
      \node[anchor=east] at (
        $0.2*(tricepsForceStart) + 0.8*(tricepsForceEnd) + (-1mm,0mm)$
      ) {\contour{mittelblau!10}{\textcolor{C0}{$\forceT$}}};
      \node[anchor=west] at (
        $0.2*(bicepsForceStart) + 0.8*(bicepsForceEnd) + (1mm,0mm)$
      ) {\contour{mittelblau!10}{\textcolor{C0}{$\forceB$}}};
      \node[anchor=west] at (
        $0.2*(loadForceStart) + 0.8*(loadForceEnd) + (1mm,0mm)$
      ) {\contour{mittelblau!10}{\textcolor{C0}{$\forceL$}}};
    \end{scope}
  \end{tikzpicture}%
  \caption[Human upper limb model geometry as a raising arm movement]{%
    Human upper limb model geometry shown as a raising arm movement
    for the elbow angles
    $\theta = \ang{10}$, \ang{45}, \ang{80},
    \ang{115}, and \ang{150} \emph{(from left to right).}
    For $\theta = \ang{80}$,
    the contributing forces \emph{\textcolor{C0}{(blue)}} and
    lever arms \emph{\textcolor{C1}{(red)}} are shown.
    Taken and adapted from
    \multicite{Sprenger15Continuum,Valentin18Gradient}.%
  }%
  \label{fig:raisingArm}%
\end{figure}

\paragraph{Overall stress components}

The continuum-mechanical part of the model
is based on the theory of finite elasticity.
When muscles deform, forces act on each infinitesimally small element
of the muscles, which is known as \term{stress.}
Usually, especially in linear elasticity,
stress is measured with the \term{Cauchy stress tensor}
(also called the \term{true stress}) \cite{Soennerlind13Why}.
For non-linear stress-strain relations,
one may use other measures such as the
\term{second Piola--Kirchhoff stress}.
The second Piola--Kirchhoff stress has the additional advantage
that it is defined along the material directions,
in contrast to the Cauchy stress tensor,
which measures the stress in coordinate directions \cite{Soennerlind13Why}.

In \multicite{Sprenger15Continuum,Roehrle16Two,Valentin18Gradient},
the strain energy function is defined such that the
resulting overall second Piola--Kirchhoff stress $\mat{S}_\mathrm{MTC}$ of
the muscle-tendon complex satisfies
\begin{equation}
  \mat{S}_\mathrm{MTC}
  = \mat{S}_\mathrm{iso} + \mat{S}_\mathrm{aniso} - p \mat{C}^{-1},
\end{equation}
where $\mat{S}_\mathrm{iso}$ and $\mat{S}_\mathrm{aniso}$
are the \term{isotropic and anisotropic parts} of the stress, respectively,
$p$ is the \term{hydrostatic pressure} to ensure incompressibility,
and $\mat{C}$ is the \term{right Cauchy--Green deformation tensor.}
The anisotropic part $\mat{S}_\mathrm{aniso}$ is defined as
\begin{equation}
  \mat{S}_\mathrm{aniso}
  \ceq (
    \mat{S}_\mathrm{passive} +
    \act \gamma_\mathrm{M} \mat{S}_\mathrm{active}
  ) (1 - \gamma_\mathrm{ST}),
\end{equation}
cf.\ \cite{Valentin18Gradient}.
Here, $\mat{S}_\mathrm{passive}$ and $\mat{S}_\mathrm{active}$
are the \term{passive and active contributions}
due to the skeletal muscle fibers,
$\act \in \clint{0, 1}$ is the \term{activation parameter} of the
respective muscle-tendon complex, and
$\gamma_\mathrm{M}, \gamma_\mathrm{ST}$ are two \term{material parameters}
with which we can differentiate between the different types of soft tissues
of the muscle-tendon complex: fat, tendon, and muscle
\cite{Valentin18Gradient}.
Isotropic fat tissue can be obtained
by setting $\gamma_\mathrm{ST} \ceq 1$,
passive anisotropic tendon tissue
by setting $\gamma_\mathrm{ST} \ceq 0$ and $\gamma_\mathrm{M} \ceq 0$, and
skeletal muscle tissue
by setting $\gamma_\mathrm{ST} \ceq 0$ and $\gamma_\mathrm{M} \ceq 1$.
A mixture of these pure materials is
achieved by linear interpolation when setting
$\gamma_\mathrm{M}$ and $\gamma_\mathrm{ST}$ to values between zero and one
\cite{Valentin18Gradient}.
More details about the theory part of the model are given in
\multicite{Sprenger15Continuum,Roehrle16Two,Valentin18Gradient}.
